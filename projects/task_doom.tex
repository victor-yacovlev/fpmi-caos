\documentclass[russian,a4paper]{article}
\usepackage[russian]{babel}
\usepackage[utf8]{inputenc}
\usepackage{amsmath}
\usepackage{color}
\usepackage{verbatim}
\usepackage{hyperref}

\hypersetup{pdftex, 
            colorlinks=true,
            linkcolor=blue,
            citecolor=blue,
            filecolor=blue,
            urlcolor=blue,
            pdftitle=,
            pdfauthor=Alexey Salnikov,
            pdfsubject=,
            pdfkeywords=}

\pagestyle{empty}

\textwidth=17cm
\hoffset=0cm
\voffset=0cm
\headheight=0cm
\topmargin=0cm
\oddsidemargin=0cm


\title{Задание практикума: сетевая игра, консольный Doom}
\author{Алексей Сальников}
\date{}

\begin{document}

\maketitle

%
% Вариант 1
%
\section{Общее описание}

Требуется реализовать игровой сервер, игровой клиент, отображатель статистики.
Сервер и клиенты должны взаимодействовать через сеть путём установки TCP
соединения.

При своём запуске сервер читает файл с картой (имя файла указывается в
параметрах при запуске). Сервер должен быть реализован как демон в
Unix, тем самым вся диагностическая информация должна выдаваться в
специальный файл журнала (Можно в syslog). Для имени файла карты и 
файла журнала в программе должны быть предусмотрены имена по умолчанию, 
которые будут подставлены если имена файлов не были указаны в командной 
строке при запуске. 
Также специальными параметрами указываются номер TCP порта 
и имя файла, где будет записан pid сервера (это необходимо чтобы можно 
было посылать сигналы именно этому экземпляру сервера), 
по умолчанию: /var/run/как\_там\_мой\_сервер\_называется). Также 
необходимо иметь возможность запустить серевер не как демон, 
а как обычную программу.

Клиент соединятся с сервером (hostname сервера и номер TCP порта
сервера указываются клиенту в аргументах main при запуске
(если номер порта небыл указан, использовать некоторый номер порта по
умолчанию)). Клиент должен отображать игровое пространство, позволять
игроку делать ход, выходить из игры по желанию игрока. (В том числе
корректно завершаться по нажатию на Ctrl+c и Ctrl+d).

%Отображатель статистики, это отдельная программа, которая может 
%обращаться к серверу с применением IPC и показывать сведения о ведущих-ся 
%на сервере играх.

Игроки на сервере могут выступать в 2-х ролях: в роли создателя команды игроков,
в роли участника команды игроков. В начальный момент, перед тем, как начать игру
одним из игроков создаётся команда, и он ожидает, пока нужное количество других 
участников присоединится к этой команде, чтобы начать игру. Именно создатель команды 
определяет число участников игры и момент начала игры. При подключении клиент может 
просмотреть список уже имеющихся команд. При подсоединении или создании команды 
игрок создаёт себе <<имя>> login. Размер имени не может привышать 60 символов.

\section{Описание игры}

Игра начинается после того, как создатель команды игроков объявил старт игры. 
Если игра началась, то присоединиться к ней ещё одному игроку нельзя до тех пор,
пока она не будет закончена. Создатель команды игроков может по своему желанию 
в любой момент завершить игру. Создатель не может принимать участия в игре, 
но зато по запросу может узнать координаты игроков и их уровень здоровья.

Игра происходит в прямоугольном лабиринте, представленным как набор
точек некоторой матрицы размера MxN. Лабиринт вместе с его размером
должны быть заданы в файле карты, при этом сам файл карты должен иметь
текстовое представление легко читаемое человеком. Цель игры оказаться
выжившим в лабиринте с максимальным уровнем здоровья.

Лабиринт состоит из стенок, аптечек/потравлялок, и проходов. По точке,
содержащей аптечку/потравлялку можно двигаться, по стенкам и за
границами лабиринта двигаться нельзя. Аптечка обладает
лечебным/отравляющим эффектом определённой силы. Игрок может съесть
аптечку/потравлялку, при этом к его здоровью прибавляется/отнимается
численное значение эффекта. По внешнему виду аптечки/потравлялки нельзя
ничего сказать о силе её воздействия и о знаке её воздействия. После
употребления аптечки соответствующая клетка считается просто проходом
(повторно съесть аптечку нельзя). В лабиринте могут встречаться другие
игроки, которые всегда занимают одну точку пространства (в стенке игрок
не может находиться).

Игроку сопоставляется некоторый уровень здоровья. Который с каждым
сделанным им ходом уменьшается на определённое значение (указывается в
параметрах серверу). C течением времени, если игрок не перемещается,
значение здоровья уменьшается, но не так активно, как в случае
перемещения. Игрок помещается в одну
из точек с координатами (i,j), два игрока не могут одновременно
находится в одной и той же точке. В этой точке он может видеть
состояние лабиринта на 10 точек в каждом направлении. То есть будет
виден прямоугольник с координатами (i-10,j-10,i+10,j+10). Прямоугольник
и всё что в нём есть нужно уметь показывать в консоли в текстовом
режиме.

За один ход можно произвести одну из следующих операций:
\begin{itemize}
\item съесть аптечку, находящуюся в текущей точке,
\item применить боевой заряд,
\item переместиться на 1 соседнюю клетку,
\item закладывать мину.
\end{itemize}

Боевой заряд применяется следующим образом. Вычисляется кратчайшее 
расстояние, до другого игрока по пути через клетки (если игрок оказался за
стенкой, то стенки надо обходить). Радиус действия заряда не
превышает 10. Интенсивность удара убывает с расстоянием от игрока,
который заряд применяет. Игрока, применившего заряд, удар от этого
заряда не травмирует. 

Игроку предоставляется 10 мин, каждой из которых он может заминировать
клетку в лабиринте. Мины не отображаются другим игрокам. В случае попадания 
игрока на мину к нему применяется ущерб эквивалентный применению боевого 
заряда на соседней клетке.

В случае применения заряда есть время необходимое на перезарядку,
в случае минирования игрок на определённое время обездвиживается.

В начальный момент игроки расставляются серверм на карту случайным образом. Они
должны обязательно оказаться в допустимой точке (то есть не на стенке и 
не на одной клетке с другим игроком). С начала игры объявляется 
мораторий на применение оружия определённой длительности. Длительность моратория 
задаётся в файле с картой и измеряется в секундах.

Мощность заряда и скорость убывания здоровья при движении
может быть задана в файле с картой.
Значения действий аптечек и потравлялок так же указываются в файле с картой.

\newpage
\section{Формат файла скартой}

Далее приведён пример файла с картой:
\begin{verbatim}
Map 10x20
######################
# #  #   #           #
# #   #  #  #####    #
# #####  #      #    #
#        ######### ###
#  # ##      #       #
#  ########  # ##### #
## ##        # #     #
## ### ############# #
#  #        #     #  #
#       #      #     #
######################

initial_health       = 500
hit_value            = 50
recharge_duration    = 3
mining_time          = 6
stay_health_drop     = 1
movement_health_drop = 4
step_standard_delay  = 0.1
moratory_duration    = 5

items:
1  1     10
20 20   -10
2  3     8
\end{verbatim}

В файле карты, собственно карта обрамлена
контуром из решёток, что необходимо для
наглядности.

Координаты в карте нумеруются начиная с 
единицы.

%\section{Статистика и файл журнала}
%
%при обращении из программы по сбору статистики должен выдаваться список 
%команд игроков, где для каждой команды выдаётся:
%\begin{itemize}
%\item статус игры: идёт, окончилась, не начата;
%\item дата начала игры, если начата;
%\item дата конца игры, если закончилась;  
%\item победитель, если определён;
%\item список игроков, их текущий уровень здоровья и координаты.
%\end{itemize}
%
%В файл журнала нужно сохранять сведения о начавшихся и закончившихся играх, 
%а так же выводить имена хостов подключившихся клиентов. Если клиен отключился, 
%с использованием соответствующей команды, отправляемой серверу, то факт отключения клиента, 
%так же нужно зафиксировать в журнале.

%Сервер, в случае достаточности ресурсов должен обеспечивать возможность подключения 
%до 1000 клиентов одновременно. 

\end{document}

